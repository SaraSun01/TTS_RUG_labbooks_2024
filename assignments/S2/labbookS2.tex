%%%%%%%%%%%%%%%%%%%%%%%%%%%%%%%%%%%%%%%%%%%%%%%%%%%%%%%%%%%%%%%%%%%%%%%%%%%%%%%%%%%%
% Do not alter this block (unless you're familiar with LaTeX)
\documentclass{../labbook}

%%%%%%%%%%%%%%%%%%%%%%%%%%%%%%%%%%%%%%%%%%%%%
%Fill in the appropriate information below
\lhead{Shiran Sun}
\rhead{Speech Synthesis I} 
\chead{\textbf{Lab Book S2. Due: \textbf{FRI 06.12.2024 23:59} CET}}
%%%%%%%%%%%%%%%%%%%%%%%%%%%%%%%%%%%%%%%%%%%%%

\begin{document}
\begin{mdframed}[backgroundcolor=blue!20]
\LaTeX ~submissions are mandatory (when applicable). Submitting your assignment in another format will be graded no higher than R.
\end{mdframed}

\section{Lab Book S2}
In this Lab Book you will work with verbalisation and prosody prediction form text. 

\begin{problem}{1}{10}{Verbalizing with code}

\subsubsection*{Task:}
\begin{itemize}
	\item Take the phone numbers you found in the first lab book and verbalize them with a Python code in at least two languages: English and Dutch.
For example, for English, the verbalisation of ``+31503638004'' could become "PLUS THIRTY ONE FIVE OH THREE SIX THREE EIGHT OH OH FOUR".
Note that grouping of the digits may differ: for example, the country code is often distinguished from the rest of the number:
don't read a Dutch (country code "+31") phone number +31123123112  as "PLUS THREE ONE ...".

Of course it helps to be a speaker of the language that you are developing a TTS system for, but it is not always necessary. To be able to verbalise phone numbers in Dutch, you can use existing descriptions and translations. For example, \href{https://www.dutchpod101.com/blog/2019/10/24/dutch-numbers/}{here} you can read about numbers in Dutch.

\item Think of a digit grouping that a TTS system could adopt when reading the phone numbers out loud (\href{https://youtu.be/2OX8znJHNo0?si=EG2d8ElbaailUEhU}{here} is an example of digit groupings in American English)
and introduce the phrase break and/or pause information by adding tokens <BR> for breaks and/or <PAU> for pauses where you feel necessary. 

For example, a sentence "John, please come in" could be verbalized with additional prosodic information as "JOHN <PAU> PLEASE <BR> COME IN".
You can also introduce prominence if you want to complicate your task a little bit.

\end{itemize}




Don't forget to comment in the code and explain how it works. Please motivate your choice for the digit grouping, provide references if necessary (for extensive descriptions please use this LaTeX document).

Submit the code to the GitHub repository. 
 \end{problem}

\begin{solution}
IF YOU WANT TO WRITE AN EXTENSIVE DESCRIPTION DO IT HERE. OTHERWISE USE COMMENTS FOR YOUR CODE.\\
In terms of my way of dealing with the digit grouping, I took '+31' as a whole for grouping of country codes, followed by a <PAU> tag. And then I grouped the other digits and added <BR> every 3 words, referring the example of digit groupings in AE phone numbers. 
\end{solution}

\end{document}
