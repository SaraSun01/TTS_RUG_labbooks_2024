%%%%%%%%%%%%%%%%%%%%%%%%%%%%%%%%%%%%%%%%%%%%%%%%%%%%%%%%%%%%%%%%%%%%%%%%%%%%%%%%%%%%
% Do not alter this block (unless you're familiar with LaTeX)
\documentclass{../labbook}

%%%%%%%%%%%%%%%%%%%%%%%%%%%%%%%%%%%%%%%%%%%%%
%Fill in the appropriate information below
\lhead{Shiran Sun}
\rhead{Speech Synthesis I} 
\chead{\textbf{Lab Book S1. Due: \textbf{Fri 29.11.2024 23:59} CET}}
%%%%%%%%%%%%%%%%%%%%%%%%%%%%%%%%%%%%%%%%%%%%%

\begin{document}
\begin{mdframed}[backgroundcolor=blue!20]
\LaTeX ~submissions are mandatory. Submitting your assignment in another format will be graded no higher than R.
\end{mdframed}

\section{Lab Book S1}
In this Lab Book you will work with text segmentation, the first step of converting written text to speech signal.

\begin{problem}{1}{3}{Tokenizing a sentence}

\subsubsection*{Task:}
Using the definition of a ``word'' discussed in class (based on Taylor (2009)), tokenize and verbalize (without code) the following sentences into ``words'':
\begin{itemize}
    \item Space is big
    \item You just won't believe how vastly, hugely, mind-bogglingly big it is.
    \item I'm gonna write to test@example.com if you pay me \texteuro42.42.
    \item In 1903 Polish physicist Marie Curie became the 1st woman to win a Nobel Prize, she was 36y.o. then.
\end{itemize}
\end{problem}
\begin{solution}
\begin{itemize}
    \item \texttt{<SPACE> <IS> <BIG>}
    \item \texttt{<YOU> <JUST> <WON'T> <BELIEVE> <HOW> <VASTLY> <COMMA> <HUGELY> <COMMA> <MIND-BOGGLINGLY> <BIG> <IT> <IS> }
    \item \texttt{<I'M> <GOING> <TO> <WRITE> <TO> <TEST,AT,EXAMPLE,DOT,COM> 
    <IF> <YOU> <PAY> <ME> <FORTY TWO EUROS AND FORTY TWO CENTS>}
    \item \texttt{<IN> <NINETEEN HUNDRED AND THREE> <POLISH> <PHYSICIST> <MARIE> <CURIE> <BECAME><THE> <FIRST> <WOMAN> <TO> <WIN> <A> <NOBEL> <PRIZE> <COMMA> <SHE> <WAS> <THIRTY SIX YEARS OLD> <THEN>}
\end{itemize}
\end{solution}


\begin{problem}{1}{7}{Exploring text normalization}

\subsubsection*{Preparation:}
We will adopt an ad-hoc approach, and assume that the text that we are working with is mostly natural language. So we will use regular expressions trying to find a pattern. 
Please, write your own regular expression and use \textit{only the standard/built in} regex libraries (e.g. \texttt{re} in Python). Don't use existing advanced text processing libraries such as Spacy or Keras. 

\subsubsection*{Task:}
In 'task.py' write a script with two functions that extract the following features from the text:
\begin{itemize}
    \item Function ``parse\_phone\_number'' for Dutch phone numbers in international format. They begin with a \texttt{+} sign and contain up to 12 digits separated by spaces or dashes. 
Examples are: \texttt{+31503638004, +31 50 363 80 04, +31 50-363-80-04, +31(0)58 2055 009}. Sometimes you might encounter numbers that also introduce $0$, such as \texttt{+31(0)58 2055 009}.
    \item Function ``parse\_url'' for Web links, such as \texttt{https://www.rug.nl/info/contact}, \\
        \texttt{https://en.wikipedia.org/wiki/Regular\_expression\#Syntax}, \\
        \texttt{https://www.google.com/search?q=python\%20re}.
    \end{itemize}

As always, write good comments in the code explaining how it works.
Make sure the code works "as is" and doesn't require editing paths, providing commandline argument and so on. In this assignment your code will be automatically tested for the code validity and you will be able to see the results in Actions (for your code to be considered valid each of the functions needs to succeed on one or more test cases).


\end{problem}

\end{document}
