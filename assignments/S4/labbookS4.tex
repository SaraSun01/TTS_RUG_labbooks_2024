%%%%%%%%%%%%%%%%%%%%%%%%%%%%%%%%%%%%%%%%%%%%%%%%%%%%%%%%%%%%%%%%%%%%%%%%%%%%%%%%%%%%
% Do not alter this block (unless you're familiar with LaTeX)
\documentclass{../labbook}

%%%%%%%%%%%%%%%%%%%%%%%%%%%%%%%%%%%%%%%%%%%%%
%Fill in the appropriate information below
\lhead{Group or individual}
\rhead{Speech Synthesis I}
\chead{\textbf{Lab Book S4. Due: \textbf{FRI 17.01.2025 23:59} CET}}
%%%%%%%%%%%%%%%%%%%%%%%%%%%%%%%%%%%%%%%%%%%%%
\begin{document}

%%%%%%%%%%%%%%%%%%%%%%%%%%%%%%
\section{Group members}
WRITE THE NAMES OF YOUR GROUP MEMBERS HERE
%%%%%%%%%%%%%%%%%%%%%%%%%%%%%%

\section{Lab Book S4}
\noindent 
For this labbook you will design a listening test to evaluate a TTS system. This assignment can be done in a group of 2-4 people, but you are also free to make up a "group of one" if you want to work on this assignment solo.
There are three levels of difficulty that you can select.

\bigskip

\underline{\textbf{Level 1}} [you can earn up to 7 points]. Design a listening test for a hypothetical TTS system. For this level you don't have to conduct the experiment and you don't have to synthesize speech. For illustration purposes you can use dummy set of recordings.
 
\smallskip

\noindent Your hypothetical TTS system needs to be made for a specific usage domain, for example:
\begin{itemize}
    \item TTS system for a minority language (you specify which minority language you would focus on). 
    \item TTS module for a specialized dialogue system (you specify the domain for which your dialogue system would apply).
    \item TTS system for a pathological speech (you specify pathology, whether it is oral cancer, ALS or anything else).
    \item Or your own challenging use case!
\end{itemize}

\underline{\textbf{Level 2}} [you can earn up to 8.5 points]. Replicate a listening experiment of an existing research on speech synthesis. For this level you need to search for an article that conducted a listening test to evaluate their TTS system(s) and shared the evaluation dataset. You will need to replicate one of the experiments with a different group if people, analyse the results, compare them with the original and report. 
You can search either on \href{https://scholar.google.com/}{Google Scholar} or on \href{https://paperswithcode.com/}{Papers with code} or via any other website that you find useful for searching research articles.

One example of such articles is \href{https://aclanthology.org/2022.sigul-1.3.pdf}{Do, P., Coler, M., Dijkstra, J., \& Klabbers, E. (2022, June). Text-to-speech for under-resourced languages: Phoneme mapping and source language selection in transfer learning. In Proceedings of the 1st Annual Meeting of the ELRA\/ISCA Special Interest Group on Under-Resourced Languages (pp. 16-22).}, with the stimuli published on \href{https://phat-do.github.io/sigul22/}{Phat's GitHub page}.

\smallskip

\underline{\textbf{Level 3}} [you can earn up to 10 points]. Conduct your own listening test with synthetic speech. For this level you need to formulate your own research question for a listening experiment, design this experiment and conduct it, analyse your data and report your findings. You can be creave about your research question, for example you could think about which synthetic speech would better suit some particular TTS usecases.

\begin{problem}{1}{10}{Design/conduct a listening test}

\subsubsection*{Task:}
\noindent \underline{\textbf{Level 1}}. Choose your hypothetical TTS scenario (see the examples above) and specify where and how you would use such a system. Design an evaluation protocol that would include the description of the following:

\begin{enumerate}
    \item Which aspects of synthetic speech you evaluate.
    \item What you include in the listening task.
    \item Your test design.
    \item The materials (stimuli) you use.
    \item The number and categories of listeners you recruit.
    \item How you present the stimuli and collect the listeners’ responses (to illustrate your user interface you can use https://rug.eu.qualtrics.com/ or https://www.psytoolkit.org/ or any other platform for response collection).
\end{enumerate}

Motivate your design choices.

\smallskip

\noindent \underline{\textbf{Level 2}}. Find a research article that you are going to replicate (see information above) and explain which part(s) of the listening experiment you are replicating and what results you expect to get with your listeners. Conduct your experiment and report your results. In your report you should include/reflect on the following:
\begin{enumerate}
    \item Cite the an article you are using to replicate the listening experiment. Mention which part you are replicating and whether you expect different results (and why).
    \item Set up the experiment using an experimental platform (e.g., PsyToolkit or Qualtrics). Don't forget to include a consent form and share the link to your listening test in the report.
    \item Collect responses from your listeners (ideally, aim for at least 10 people if possible).
    \item Replicate the statistical analysis (if applicable given your number of participants), compare your results with the original article and share your findings.
\end{enumerate}


\smallskip

\noindent \underline{\textbf{Level 3}}. Think of your own listening experiment (see information above), conduct it and report your results. In your report you should include/reflect on the following:
\begin{enumerate}
    \item Your research question and your hypothesis. Be creative.
    \item Create your own stimuli: either synthesize them yourself (it could be an old system, it could be the new system, it could be vocoded speech) or reuse an existing dataset. 
    \item Set up the experiment using an experimental platform (e.g., PsyToolkit or Qualtrics). Don't forget to include a consent form.
    \item Collect responses from your listeners (ideally, aim for at least 10 people if possible).
    \item Conduct the statistical analysis and share your findings (in clear connection to your research question and hypothesis).
\end{enumerate}

Motivate your choices and cite the sources you use.

\subsection*{Submission}
\noindent Submission of this group project is a LaTeX file containing the structured description of your evaluation protocol with link(s) to your experiment setting, your data analysis and findings (if applicable). Feel free to submit any additional files that you deem necessary. Don't forget to include the description of authors' contributions. One submission per group is sufficient.
\end{problem}

\begin{solution}
REPLACE THIS TEXT WITH YOUR ANSWER, OR (RE)USE A STRUCTURED ABSTRACT TEMPLATE (FROM SPEECH SOUNDS).
\end{solution}


\end{document}
