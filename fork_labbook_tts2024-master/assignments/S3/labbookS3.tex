%%%%%%%%%%%%%%%%%%%%%%%%%%%%%%%%%%%%%%%%%%%%%%%%%%%%%%%%%%%%%%%%%%%%%%%%%%%%%%%%%%%%
% Do not alter this block (unless you're familiar with LaTeX)
\documentclass{../labbook}

%%%%%%%%%%%%%%%%%%%%%%%%%%%%%%%%%%%%%%%%%%%%%
%Fill in the appropriate information below
\lhead{Shiran Sun}
\rhead{Speech Synthesis I}
\chead{\textbf{Lab Book S3. Due: \textbf{FRI 13.12.2024 23:59} CET}}
%%%%%%%%%%%%%%%%%%%%%%%%%%%%%%%%%%%%%%%%%%%%%
\begin{document}

\section{Lab Book S3}
\noindent In this lab book you will work with the two tools, {MBROLA} diphone voices with front end realized by \textbf{eSpeak}:

\noindent Dutoit, T., Pagel, V., Pierret, N., Bataille, F., \& Van der Vrecken, O. (1996, October). The MBROLA project: Towards a set of high quality speech synthesizers free of use for non commercial purposes. In Proceeding of Fourth International Conference on Spoken Language Processing. ICSLP'96 (Vol. 3, pp. 1393-1396). IEEE.

\noindent \href{https://espeak.sourceforge.net/}{https://espeak.sourceforge.net/}


\begin{problem}{2}{7}{Manually managing code-switching with eSpeak and MBROLA}
\subsubsection*{Intro:}
eSpeak is a compact open source software speech synthesizer for English and other languages, that uses a "formant synthesis" method. This allows many languages to be provided in a small size. 
We will use eSpeak-NG, a fork of original eSpeak (and call it just "eSpeak" for brevity) as a front end for MBROLA diphone voices (which are cost-free but not open source). We will be working with the \texttt{us1} voice.

\subsection*{Installing and trying out MBROLA and eSpeak}
Both MBROLA and eSpeak-NG can be installed out of the box. For Debian-based Linuxes you can use 
\begin{verbatim} 
    apt install mbrola-us1
    apt install espeak-ng
\end{verbatim} (if you are using Google colab, prefix the shell commands with an exclamation mark, e.g. 
\texttt{!apt install mbrola-us1}).
\smallskip

You can find different commands for eSpeak here: \href{https://espeak.sourceforge.net/commands.html}{https://espeak.sourceforge.net/commands.html}.
For the following tasks we will need the transcription that eSpeak generates based on the available diphones for the US1 voice \footnote{There are problems with phoneme ``02'', equivalent to /\textipa{6}/ in “often” and “software” in MBROLA, based on the documentation there are at least 2 similar sounds you can consider instead: /\textipa{2}/ (MBROLA symbol “V”) as in “nut” (in the original MBROLA transcription this sound is used to synthesize the vowel in “of”), or /\textipa{A}/ (MBROLA symbol “A”) as in "Arthur"}. 
Here are commands that we will use:
\begin{itemize}
    \item To examine a transcription you can call 
        \begin{verbatim} 
            espeak-ng -v mb-us1 -q --pho 'Hey! Enter your text here'
        \end{verbatim} where 
        \begin{itemize}
            \item \texttt{-v mb-us1} asks to use the "us1" voice from mbrola
            \item \texttt{-q} disables sound output
            \item  \texttt{--pho} prints the transcription (in this case -- in mbrola format)
        \end{itemize}
                
      
      The transcription consists of multiple lines similar to \texttt{EI	60	 0 232 80 215 100 215}:
      \begin{itemize}
          \item \texttt{EI} is the phoneme to be generated
          \item \texttt{60} is its duration (60ms)
          \item the rest is a sequence of (time-F0) pairs, where time is in percents of the phoneme's duration. In this example, it starts with F0=232, changes to 215 at T=80\% (thus, 48ms) and then to 251 Hz at the end of the phoneme.
      \end{itemize}

    \item To save this transcription to a file "testmb":
          \begin{verbatim}
          espeak-ng -v mb-us1 -q --pho 'Hey there' > testmb  
          \end{verbatim}
    \item To synthesize speech based on the transcription in "testmb" file using          US1 voice and to write it into a "test.wav" file:
          \begin{verbatim}
           mbrola /usr/share/mbrola/us1/us1 testmb test.wav   
          \end{verbatim}
\end{itemize}
    
\subsubsection*{Task:}
\noindent Here is a short text in English that uses Dutch names: 

\textbf{\texttt{Characters from Dutch children's books - Jip and Janneke - has recently become part of an expression "jip-en-janneketaal" meaning "simple language" or layman's terms. It is most often used in context of politics and software engineering.}}
\smallskip

You will need to:
\begin{itemize}
    \item Transcribe the text with eSpeak.
    \item Synthesize spoken version of this text using MBROLA's US1 voice.
    \item Edit the transcription manually to adjust to pronunciation of Dutch words in English text so they will sound as Dutch as possible given the US1 set of diphones \footnote{"jip-en-janneketaal" in IPA is \textipa{[jIp En jAn@k@tA:l]}}.
    \item Synthesize the spoken version of this text based on the adjusted transcription using MBROLA's US1 voice.
    \item Reflect on the results.
    \item Describe the steps you had to take: why you changed any of diphones (if any) and why you adjusted duration and pitch settings for certain diphones (if any). 
\end{itemize}
\subsection*{Submission}
\noindent Alongside with this LaTeX file with your steps description and reflections, commit the adjusted transcription and the new voice sample(s).
\end{problem}

\begin{solution}
\begin{itemize}
    \item Reflect on the results.\\
    After manually adjusting the phoneme file for "Jip-en-Janneketaal", the resulting pronunciation sounded closer to Dutch, especially compared to the original English-oriented phoneme file. The adjustments improved the overall Dutch-like quality of the speech synthesis, particularly the diphthongs and vowels that differ significantly between English and Dutch.\\
    However, some minor issues persisted:\\
    1.The ‘@’ still sounded slightly different because the MBROLA US1 voice is designed for American English.\\
    2.'A' as a long vowel didn't fully match the Dutch /\textipa{a:}/ sound, though the adjustment helped.\\
    3.The slowed-down speech rate (-t 1.5) made individual phonemes clearer, helping achieve more precise pronunciation.\\
\end{itemize}

\begin{itemize}
   \item Describe the Steps Taken.\\
    a. Phoneme Adjustments:\\
    1.Replaced ‘dZ' with ‘j':The original used 'dZ',representing the English ‘J' as in ‘Jim.' Since Dutch ‘Jip' uses /j/, ‘j' was the correct phoneme.\\
    2.Replaced ‘I' with ‘@' (Schwa): The ‘e' in ‘Janneke' is a schwa , so ‘@' replaced ‘I'.\\
    3.Corrected ‘Taal' Vowel: Kept the final vowel ‘A' of ‘taal', attempting to approximate /{\textipa{a:}}/.\\
    4.Removed 4: ‘t' replaced ‘4'.\\
    b.Duration and Pitch Adjustments\\
    Slowing Down Speech (-t 1.5): To ensure clearer articulation and allow phonemes like vowels to be distinguished, speech rate was slowed by 1.5 times using MBROLA’s -t flag.
\end{itemize}
\end{solution}

\begin{problem}{2}{3}{Automatic mapping of transcriptions}
\subsubsection*{Task:}
\noindent Imagine you have a Dutch transcription made for Dutch MBROLA voices. Unfortunately, the diphone database is no longer available for you, so you need to make an automatic mapping of Dutch eSpeak transcription to English eSpeak transcription so MBROLA can read it with US1 voice.
\begin{itemize}
    \item Take the following text "Ik wil wel graag werken aan dit project, maar ik weet niet of het gaat lukken. Ik zie veel beren op de weg."\footnote{Translation: "I would like to work on this project, but I don't know if it will work. I see bumps in the road ahead" (literal translation of "ik zie veel beren op de weg" is "I see a lot of bears on the road").}
    \item To generate Dutch transcription, use \texttt{nl2} voice \footnote{Don't forget to install the \texttt{mbrola-nl2} package} (the \texttt{nl1} one seems to be broken).
    \item Similar to the previous task, describe the steps you had to take: what are the changes you decided to introduce and why. 
\end{itemize}

\subsection*{Submission}
\noindent Alongside with this LaTeX file with your reflections, commit the complete code you used, the transcriptions - original and resulting, and the voice sample that you get by using the new transcription. Don't forget to write clear comments in your code.
\end{problem}

\begin{solution}
After generating the Dutch phoneme transcription using nl2, adjustments were made to adapt the phonemes for compatibility with the MBROLA US1 voice, which supports English phonemes. Below are the mapping steps, reasons for changes, and relevant phonetic explanations:\\
a. Phoneme Adjustments:\\
1.A \textipa{/a/}→A: \textipa{/a:/}\\
Shorter in Dutch, longer in English\\
2.e/e:/→EI/e\textipa{I}/\\
Dutch /e:/ are close to the English diphthong /e\textipa{I}/as in “day.” This adjustment helps maintain the intended rising vowel quality.\\
3.r/r/→r=/\textipa{R}/\\
 Dutch /r/ is typically rolled or tapped. Since MBROLA doesn’t support a trilled /r/, the English approximant/\textipa{R}/ was used to keep the pronunciation natural.\\
4.x/\textipa{X}/→k/k/\\
Dutch /\textipa{X}/(voiceless velar fricative) is not present in MBROLA's phoneme set. The closest alternative is the voiceless stop /k/, even though the articulation point differs.\\
5.Y→U\\
Dutch Y is a rounded front vowel similar to English U as in “book”. Though not identical, U is the closest English equivalent in the MBROLA system.\\
6.G/\textipa{G}/→g/g/\\
Dutch /\textipa{G}/ (voiced velar fricative) is not supported in MBROLA. The voiced velar stop /g/ was chosen as a suitable alternative for clarity and consistency.\\
7.L/l/→l/l/\\
Dutch /l/ was mapped to English /l/ since MBROLA supports this phoneme without issue.\\
8.a/a/→\{ /æ/\\
Dutch /a/ is closer to the English /æ/ as in “cat”. This change also ensures that the vowel remains recognizable in MBROLA’s system.\\
9.o→V (Task 1 Adjustment)\\
 b.Duration and Pitch Adjustments\\
    Slowing Down Speech (-t 1.5): To ensure clearer articulation and allow phonemes like vowels to be distinguished, speech rate was slowed by 1.5 times using MBROLA’s -t flag.



\end{solution}

\end{document}
